\documentclass[10pt,spanish]{beamer}
\usepackage[spanish]{babel}
\usepackage{graphicx}
\usepackage{amsmath}
\usepackage{booktabs}
\usepackage{tabularx}
\usepackage{hyperref}
\usepackage{listings}
\usepackage{xcolor}

% Tema de Beamer
\usetheme{Madrid}
\usecolortheme{default}

% Definimos colores estilo "terminal con fondo negro"
\definecolor{backcolour}{rgb}{0.1,0.1,0.1}
\definecolor{codegreen}{rgb}{0,0.8,0}
\definecolor{codegray}{rgb}{0.7,0.7,0.7}
\definecolor{codepurple}{rgb}{0.8,0.6,1}
\definecolor{codewhite}{rgb}{1,1,1}

\lstdefinestyle{mypython}{
    backgroundcolor=\color{backcolour},
    basicstyle=\ttfamily\scriptsize\color{codewhite},
    commentstyle=\color{codegreen},
    keywordstyle=\color{codepurple},
    stringstyle=\color{codegreen},
    numbers=left,
    numberstyle=\tiny\color{codegray},
    breaklines=true,
    breakatwhitespace=false,
    showstringspaces=false,
    tabsize=4
}

\lstset{style=mypython}

% Información del título
\title{Clasificación de Riesgo Cardiovascular}
\subtitle{Proyecto 2: Introducción a la Ciencia de Datos}
\author{Hazel Shamed Sánchez Chávez}
\institute{Centro de Investigación en Matemáticas (CIMAT)}
\date{}
\titlegraphic{\includegraphics[width=3cm]{figures/logo_cimat.png}}

\begin{document}

\begin{frame}
    \titlepage
\end{frame}

\begin{frame}{Contenido}
    \begin{itemize}
        \item Introducción
        \item Exploración de datos
        \item Preprocesamiento
        \item Modelos de Clasificación
        \item Resultados y Comparación
        \item Conclusiones
    \end{itemize}
\end{frame}

\section{Introducción}
\begin{frame}{Introducción}
    \begin{itemize}
        \item Las enfermedades cardiovasculares (ECV) representan \textbf{32\% de todas las defunciones} globales (OMS)
        \item Identificación temprana: desafío crítico para sistemas de salud
        \item Dataset \textit{"Heart Disease"} del repositorio UCI:
        \begin{itemize}
            \item 303 pacientes
            \item Registros clínicos y biomédicos
            \item Signos clínicos, parámetros de laboratorio, características electrocardiográficas
        \end{itemize}
        \item \textbf{Objetivo}: Clasificador binario para discriminar presencia/ausencia de enfermedad cardiovascular
    \end{itemize}
\end{frame}

\section{Exploración de datos}
\begin{frame}{Base de datos Heart Disease}
    \begin{itemize}
        \item \textbf{Fuente}: Cleveland Clinic Foundation (Dr. Robert Detrano)
        \item \textbf{Período}: 1979-1988 (10 años)
        \item \textbf{Pacientes}: 303 referidos para evaluación coronaria
        \item \textbf{Criterio diagnóstico}: Angiografía coronaria ($\geq$50\% de estrechamiento)
        \item \textbf{Dimensión}: 303 filas $\times$ 13 columnas + etiqueta
    \end{itemize}
\end{frame}

\begin{frame}{Variables del Dataset}
    \scriptsize
    \begin{table}
        \begin{tabular}{lllp{4cm}}
            \toprule
            \textbf{Variable} & \textbf{Tipo} & \textbf{Descripción} \\
            \midrule
            age & Continuo & Edad en años (29-77) \\
            sex & Categórico & Sexo (0=fem, 1=masc) \\
            cp & Categórico & Tipo de dolor torácico (1-4) \\
            trestbps & Continuo & Presión arterial en reposo (mm Hg) \\
            chol & Continuo & Colesterol sérico (mg/dl) \\
            fbs & Categórico & Glicemia en ayunas $>$120 mg/dl \\
            restecg & Categórico & Resultado ECG en reposo \\
            thalach & Continuo & Frecuencia cardíaca máxima \\
            exang & Categórico & Angina inducida por ejercicio \\
            oldpeak & Continuo & Depresión ST inducida \\
            slope & Categórico & Pendiente del segmento ST \\
            ca & Continuo & Vasos principales coloreados \\
            thal & Categórico & Tipo de defecto talio \\
            \bottomrule
        \end{tabular}
    \end{table}
\end{frame}

\begin{frame}{Estadísticas Descriptivas}
    \scriptsize
    \begin{table}
        \begin{tabular}{l*{6}{r}}
            \toprule
            & \textbf{age} & \textbf{trestbps} & \textbf{chol} & \textbf{thalach} & \textbf{oldpeak} & \textbf{ca} \\
            \midrule
            count & 303 & 303 & 303 & 303 & 303 & 299 \\
            mean & 54.4 & 131.69 & 246.69 & 149.61 & 1.04 & 0.67 \\
            std & 9.04 & 17.60 & 51.78 & 22.88 & 1.16 & 0.94 \\
            min & 29 & 94 & 126 & 71 & 0.00 & 0 \\
            max & 77 & 200 & 564 & 202 & 6.20 & 3 \\
            \bottomrule
        \end{tabular}
    \end{table}
    
    \begin{itemize}
        \item \textbf{Distribución objetivo}: 164 sin enfermedad vs 139 con enfermedad
        \item \textbf{Valores faltantes}: en variables \texttt{ca} y \texttt{thal}
        \item \textbf{Consideraciones}: Datos históricos (1979-1988), sesgo de selección
    \end{itemize}
\end{frame}

\section{Preprocesamiento}
\begin{frame}{Preprocesamiento de Datos}
    \begin{block}{Objetivos}
        \begin{itemize}
            \item Homogenizar escala de variables numéricas
            \item Separar predictores $X$ y respuesta $y$
            \item Manejar valores faltantes y outliers
        \end{itemize}
    \end{block}

    \begin{columns}
        \column{0.5\textwidth}
        \centering
        \includegraphics[width=0.8\textwidth]{figures/ca_thal_nans.png}
        
        \column{0.5\textwidth}
        \centering
        \includegraphics[width=0.8\textwidth]{figures/ca_thal_boxplots.png}
    \end{columns}
    
    \begin{itemize}
        \item \textbf{Valores faltantes}: Eliminación de filas (297 observaciones finales)
        \item \textbf{Escalado}: StandardScaler (media=0, varianza=1)
    \end{itemize}
\end{frame}

\section{Modelos de Clasificación}
\begin{frame}{Random Forest}
    \begin{columns}
        \column{0.6\textwidth}
        \begin{itemize}
            \item Combina múltiples árboles de decisión
            \item Maneja variables categóricas/continuas
            \item Captura relaciones no lineales
            \item Proporciona importancia de variables
        \end{itemize}
        
        \column{0.4\textwidth}
        \centering
        \includegraphics[width=\textwidth]{figures/feature_importance.png}
    \end{columns}
\end{frame}

\begin{frame}{Regresión Logística}
    \begin{columns}
        \column{0.6\textwidth}
        \begin{itemize}
            \item Modelo lineal interpretable
            \item Coeficientes como \textit{odds ratios}
            \item Bajo riesgo de sobreajuste
            \item Supone linealidad en el logit
        \end{itemize}
        
        \column{0.4\textwidth}
        \centering
        \includegraphics[width=\textwidth]{figures/logistic_coefficients.png}
    \end{columns}
\end{frame}

\begin{frame}{Red Neuronal}
    \begin{columns}
        \column{0.6\textwidth}
        \begin{itemize}
            \item Aprende patrones no lineales complejos
            \item Flexible con arquitecturas específicas
            % \item Regularización con \textit{dropout}
            \item Requiere más datos y ajuste cuidadoso
        \end{itemize}
        
        \column{0.4\textwidth}
        \centering
        \includegraphics[width=\textwidth]{figures/loss_function.png}
    \end{columns}
\end{frame}

\section{Resultados y Comparación}
\begin{frame}{Métricas de Desempeño}
    \scriptsize
    \begin{table}
        \begin{tabular}{lccc}
            \toprule
            \textbf{Métrica} & \textbf{Random Forest} & \textbf{Logistic Regression} & \textbf{Neural Network} \\
            \midrule
            Recall & 0.750 & 0.786 & \textbf{0.857} \\
            Accuracy & 0.817 & 0.833 & \textbf{0.850} \\
            Balanced Accuracy & 0.812 & 0.830 & \textbf{0.850} \\
            AUC-ROC & 0.941 & \textbf{0.950} & 0.910 \\
            Precision & 0.840 & \textbf{0.846} & 0.828 \\
            F1-score & 0.792 & 0.815 & \textbf{0.842} \\
            \bottomrule
        \end{tabular}
    \end{table}
    
    \begin{itemize}
        \item \textbf{Red Neuronal}: Mejor recall y accuracy (minimiza falsos negativos)
        \item \textbf{Regresión Logística}: Mejor AUC-ROC (capacidad discriminativa global)
        \item \textbf{Random Forest}: Posición intermedia en todas las métricas
    \end{itemize}
\end{frame}

\begin{frame}{Matrices de Confusión}
    \centering
    \includegraphics[width=0.7\textwidth]{figures/confusion_matrix.png}
    
    \begin{itemize}
        \item Red neuronal detecta mejor casos positivos (menos falsos negativos)
        \item Patrones de error diferentes entre modelos
    \end{itemize}
\end{frame}

\section{Conclusiones}
\begin{frame}{Conclusiones}
    \begin{block}{Hallazgos Principales}
        \begin{itemize}
            \item \textbf{Regresión Logística}: Mejor capacidad discriminativa global (AUC-ROC 0.950)
            \item \textbf{Red Neuronal}: Mayor sensibilidad (recall 0.857) - ideal para minimizar falsos negativos
            \item \textbf{Random Forest}: Desempeño sólido pero intermedio
        \end{itemize}
    \end{block}
    
    \begin{block}{Recomendaciones Clínicas}
        \begin{itemize}
            \item Prioridad en detección: Red Neuronal
            \item Equilibrio y interpretabilidad: Regresión Logística
            \item Alternativa robusta: Random Forest
        \end{itemize}
    \end{block}
\end{frame}

\begin{frame}{Limitaciones y Trabajo Futuro}
    \begin{block}{Limitaciones}
        \begin{itemize}
            \item Datos históricos (1979-1988)
            \item Sesgo de selección (pacientes referidos)
            \item Valores faltantes en variables clave
            \item Desbalance de clases moderado
        \end{itemize}
    \end{block}
    
    \begin{block}{Trabajo Futuro}
        \begin{itemize}
            \item Incorporar factores de riesgo modernos
            \item Validación en poblaciones contemporáneas
            \item Análisis de características por subpoblaciones
            \item Implementación en entorno clínico real
        \end{itemize}
    \end{block}
\end{frame}

\begin{frame}{Referencias}
    \scriptsize
    \begin{itemize}
        \item \textbf{Dataset}: Detrano, R. et al. (1988). \textit{Heart Disease Dataset}. UCI Machine Learning Repository.
        
        \item \textbf{Organización Mundial de la Salud (2021)}. \textit{Cardiovascular Diseases (CVDs)}.
        
        \item \textbf{Cleveland Clinic Foundation}. (1988). Multicenter cardiac study data.
        
        \item Hungarian Institute of Cardiology, University Hospital Zurich, University Hospital Basel. Contribuidores de datos.
    \end{itemize}
    
    \vspace{0.5cm}
    \centering
    \textbf{¡Gracias por su atención!}
\end{frame}

\end{document}