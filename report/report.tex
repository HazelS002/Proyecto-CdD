\documentclass[10pt]{article}
\usepackage[spanish]{babel}
\usepackage[a4paper, tmargin=0.75in, lmargin=0.80in, rmargin=0.80in,
    bmargin=1in]{geometry}
\usepackage{hyperref}
%\usepackage{multicol}
\hypersetup{
    colorlinks=true,
    linkcolor=black,
    filecolor=magenta,      
    urlcolor=blue,
    citecolor=black,
}
%\usepackage[numbers,sort&compress]{natbib} % for a numerical citation list
\usepackage{natbib} % to cite references by surname and year
\usepackage{graphicx}
\usepackage{amsfonts}
\usepackage{amsthm}
\usepackage{amssymb}
\usepackage{lipsum}
\usepackage{amsmath}
\usepackage{tabularx}
\usepackage{pdflscape}
\usepackage{booktabs}
\usepackage{bbm}
\usepackage{listings}
\usepackage{xcolor}

\usepackage[
  backend=biber,
  style=alphabetic,
  citestyle=alphabetic,
  doi=true,
  url=true,
  isbn=false,
  eprint=false,
  maxbibnames=99
]{biblatex}

\addbibresource{references.bib}

% Definimos colores estilo "terminal con fondo negro"
\definecolor{backcolour}{rgb}{0.1,0.1,0.1}
\definecolor{codegreen}{rgb}{0,0.8,0}
\definecolor{codegray}{rgb}{0.7,0.7,0.7}
\definecolor{codepurple}{rgb}{0.8,0.6,1}
\definecolor{codewhite}{rgb}{1,1,1}


\lstdefinestyle{mypython}{
    backgroundcolor=\color{backcolour},
    basicstyle=\ttfamily\scriptsize\color{codewhite},
    commentstyle=\color{codegreen},
    keywordstyle=\color{codepurple},
    stringstyle=\color{codegreen},
    numbers=left,
    numberstyle=\tiny\color{codegray},
    breaklines=true,
    breakatwhitespace=false,
    showstringspaces=false,
    tabsize=4
}

\lstset{style=mypython}


\pagestyle{empty}


%%%%%%%%%%%%%%%%%%%%%%%%%%%%%%%%%%%%%%%%%%%%%%%%%%
%%%%%%%%%%%%%%%%%%%%%%%%%%%%%%%%%%%%%%%%%%%%%%%%%%
% ENTER SOME IMPORTANT INFORMATION
%%%%%%%%%%%%%%%%%%%%%%%%%%%%%%%%%%%%%%%%%%%%%%%%%%
%%%%%%%%%%%%%%%%%%%%%%%%%%%%%%%%%%%%%%%%%%%%%%%%%%
\newcommand{\studentname}{Hazel Shamed Sánchez Chávez}
% \newcommand{\researchcentre}{Maestría en Probabilidad y Estadística}
\newcommand{\institution}{Centro de Investigación en Matemáticas (CIMAT)}
\newcommand{\projecttitle}{Proyecto: Clasificación de Riesgo Cardiovascular}
\newcommand{\supervisor}{Dr. Marco Antonio Aquino López}
%%%%%%%%%%%%%%%%%%%%%%%%%%%%%%%%%%%%%%%%%%%%%%%%%%
%%%%%%%%%%%%%%%%%%%%%%%%%%%%%%%%%%%%%%%%%%%%%%%%%%

\setlength{\parindent}{1em}
\setlength{\parskip}{1em}


\begin{document}

    \begin{center}
        {\Large{Proyecto 2: Clasificación de Riesgo Cardiovascular}} \\
        \vspace{2mm}
        {\Large{Introducción a la Ciencia de Datos}} \\
    \end{center}

    \vspace{5mm}
    \hrule
    \vspace{1mm}
    \hrule

    \vspace{3mm}
    \begin{tabular}{ll} 
        Estudiante:           	        & {\studentname}   \\ 
        % Programa Educativo: 	        & {\researchcentre}  \\ 
        Institución:                 & {\institution}  \\
        Profesor: 	                 & {\supervisor}  \\ 
    \end{tabular}

    \vspace{3mm}
    \hrule
    \vspace{1mm}
    \hrule

    % \begin{abstract}
    % \end{abstract}

    %%%%%%%%%%%%%%%%%%%%%%%%%%%%%%%%%%%%%%%%%%%%%%%%%%%%%%%%%%%%%%%%%%%%%%%%%%%%
    %%%%%%%%%%%%%%%%%%%%%%%%%%% INTRODUCCIÓN %%%%%%%%%%%%%%%%%%%%%%%%%%%%%%%
    %%%%%%%%%%%%%%%%%%%%%%%%%%%%%%%%%%%%%%%%%%%%%%%%%%%%%%%%%%%%%%%%%%%%%%%%%%%%

    \section{Introducción}

    En el ámbito de la salud pública, las enfermedades
    cardiovasculares (ECV) constituyen una de las principales causas de
    mortalidad a nivel global, representando aproximadamente el 32\% de todas
    las defunciones según la Organización Mundial de la Salud. La identificación
    temprana de individuos en riesgo representa un desafío crítico para los
    sistemas de salud, con implicaciones significativas en la reducción de la
    carga asistencial y la mejora de resultados clínicos.

    Este proyecto se enfoca en el desarrollo de un modelo predictivo de
    clasificación de riesgo cardiovascular utilizando el dataset
    \textit{"Heart Disease"} del repositorio UCI Machine Learning, que contiene
    registros clínicos y biomédicos de 303 pacientes. La base de datos integra
    signos clínicos, parámetros de laboratorio y características
    electrocardiográficas, proporcionando una base multidimensional para el
    análisis.

    Las ECV representan condiciones patológicas que afectan el corazón y los
    vasos sanguíneos, incluyendo enfermedad coronaria, enfermedad
    cerebrovascular y cardiopatía reumática. La naturaleza multifactorial de
    estas patologías demanda aproximaciones analíticas capaces de integrar
    diversos predictores, desde factores de riesgo convencionales hasta
    marcadores específicos de función cardíaca.

    El objetivo central de este proyecto es construir un clasificador
    binario que permita discriminar entre pacientes con presencia y ausencia de
    enfermedad cardiovascular, utilizando únicamente variables accesibles
    mediante exámenes clínicos rutinarios. Este enfoque pretende superar las
    limitaciones de los métodos diagnósticos invasivos, optimizando la
    asignación de recursos diagnósticos especializados.

    La implementación exitosa de este modelo podría servir como herramienta de
    apoyo a la decisión clínica, facilitando la estratificación prioritaria de
    pacientes y contribuyendo a estrategias de prevención secundaria más
    efectivas en el manejo de enfermedades cardiovasculares.


    \section{Exploración inicial de los datos}\label{sec:dataset}

    \subsection{Base de datos \textit{Heart Disease}.}


    El dataset ``Heart Disease'', de $303$ filas por $13$ columnas (mas etiquetas),
    fue recopilado principalmente por la \textbf{Cleveland Clinic Foundation}
    bajo la dirección del Dr. Robert Detrano, M.D., Ph.D., con la colaboración
    de múltiples instituciones internacionales incluyendo el Hungarian Institute
    of Cardiology de Budapest, University Hospital de Zurich y University
    Hospital de Basel. Los datos fueron donados al UCI Machine Learning
    Repository en 1988, donde están disponibles públicamente con el
    identificador $45$.

    El estudio se condujo como una investigación observacional multicéntrica
    entre 1979 y 1988, abarcando aproximadamente 10 años de recolección de
    datos. La población del estudio consistió en 303 pacientes referidos para
    evaluación coronaria en las instituciones participantes. El criterio de
    referencia para el diagnóstico fue la angiografía coronaria, definiendo como
    caso positivo la presencia de $\geq$50\% de estrechamiento diametral.

    El dataset original contenía 76 variables recolectadas, pero para el
    análisis estándar se seleccionaron 14 atributos que incluyen signos
    clínicos, parámetros de laboratorio y características
    electrocardiográficas.
  
    \subsection{Variables del Dataset}

    Al descargar el dataset, las variables categoricas ya vienen dadas de forma
    numerica, asignando un valor entero a cada categoría de cada variable. Esto
    facilita la manipulación y calculos de los datos. Presentamos una tabla
    resumen de las variables del dataset:

    \begin{table}[ht]
        \centering
        \small
        \caption{Variables del Dataset Heart Disease}
        \label{tab:variables}
        \begin{tabular}{>{\raggedright\arraybackslash}p{2cm}>{\raggedright\arraybackslash}p{2cm}>{\raggedright\arraybackslash}p{6cm}>{\raggedright\arraybackslash}p{3cm}}
            \toprule
            \textbf{Atributo} & \textbf{Tipo} & \textbf{Descripción} & \textbf{Valores} \\
            \midrule
            age & Continuo & Edad en años & 29-77 \\
            sex & Categórico & Sexo & 0 = fem, 1 = masc \\
            cp & Categórico & Tipo de dolor torácico & 1-4 \\
            trestbps & Continuo & Presión arterial en reposo (mm Hg) & 94-200 \\
            chol & Continuo & Colesterol sérico (mg/dl) & 126-564 \\
            fbs & Categórico & Glicemia en ayunas $>$120 mg/dl & 0 = no, 1 = sí \\
            restecg & Categórico & Resultado ECG en reposo & 0-2 \\
            thalach & Continuo & Frecuencia cardíaca máxima alcanzada & 71-202 \\
            exang & Categórico & Angina inducida por ejercicio & 0 = no, 1 = sí \\
            oldpeak & Continuo & Depresión ST inducida por ejercicio & 0-6.2 \\
            slope & Categórico & Pendiente del segmento ST en ejercicio & 1-3 \\
            ca & Continuo & Número de vasos principales coloreados & 0-3 \\
            thal & Categórico & Tipo de defecto talio & 3 = normal, 6 = defecto, 7 = reversible \\
            \midrule
            num & Objetivo & Diagnóstico de enfermedad & 0 = no, 1-4 = sí \\
            \bottomrule
        \end{tabular}
    \end{table}

    Estos signos clínicos, parámetros de laboratorio y características
    electrocardiográficas se interpretan desde el punto de vista médico de la
    siguiente manera:
    \begin{itemize}
        \item El dolor torácico (cp) se clasifica en cuatro tipos: anginoso
        típico (dolor opresivo por esfuerzo), atípico, no anginoso y ausente.
        \item El colesterol sérico (chol) cuantifica los lípidos en sangre en
        $mg/dl$.
        \item La glicemia en ayunas (fbs) indica diabetes si supera $120 mg/dl$.
        \item El electrocardiograma en reposo (restecg) detecta anomalías ST-T o
        hipertrofia ventricular.
        \item La frecuencia cardíaca máxima (thalach) registra el pico de
        esfuerzo.
        \item La angina inducida por ejercicio (exang) señala dolor durante la
        prueba de esfuerzo.
        \item La depresión del segmento ST (oldpeak) mide en milivolts la
        isquemia miocárdica durante el ejercicio.
        \item La pendiente del segmento ST (slope) evalúa si es ascendente
        (normal), plana o descendente (patológica).
        \item Los vasos principales coloreados (ca) indican cuántas arterias
        coronarias tienen obstrucción ≥50\% en angiografía.
        \item La prueba de talio (thal) evalúa perfusión cardíaca: normal,
        defecto fijo (infarto) o reversible (isquemia).
        \item El diagnóstico (num) clasifica la severidad de la enfermedad
        desde $0$ (ausente) hasta $4$ (severa).
    \end{itemize}
    
    A continuación, se presentan las estadísticas descriptivas de estas 
    variables:

    \newpage

    \begin{table}[ht]
        \centering
        \small
        \caption{Estadísticas Descriptivas del Dataset}
        \label{tab:estadisticas_descriptivas}
        \begin{tabular}{l*{13}{r}}
            \toprule
            & \textbf{age} & \textbf{sex} & \textbf{cp} & \textbf{trestbps} & \textbf{chol} & \textbf{fbs} & \textbf{restecg} & \textbf{thalach} & \textbf{exang} & \textbf{oldpeak} & \textbf{slope} & \textbf{ca} & \textbf{thal} \\
            \midrule
            count & 303 & 303 & 303 & 303 & 303 & 303 & 303 & 303 & 303 & 303 & 303 & 299 & 301 \\
            mean & 54.4 & 0.68 & 3.16 & 131.69 & 246.69 & 0.15 & 0.99 & 149.61 & 0.33 & 1.04 & 1.60 & 0.67 & 4.73 \\
            std & 9.04 & 0.47 & 0.96 & 17.60 & 51.78 & 0.36 & 0.99 & 22.88 & 0.47 & 1.16 & 0.62 & 0.94 & 1.94 \\
            min & 29 & 0 & 1 & 94 & 126 & 0 & 0 & 71 & 0 & 0.00 & 1 & 0 & 3 \\
            25\% & 48 & 0 & 3 & 120 & 211 & 0 & 0 & 134 & 0 & 0.00 & 1 & 0 & 3 \\
            50\% & 56 & 1 & 3 & 130 & 241 & 0 & 1 & 153 & 0 & 0.80 & 2 & 0 & 3 \\
            75\% & 61 & 1 & 4 & 140 & 275 & 0 & 2 & 166 & 1 & 1.60 & 2 & 1 & 7 \\
            max & 77 & 1 & 4 & 200 & 564 & 1 & 2 & 202 & 1 & 6.20 & 3 & 3 & 7 \\
            \bottomrule
        \end{tabular}
    \end{table}

    \subsection{Consideraciones Adicionales}

    Es importante señalar que el dataset presenta algunos valores faltantes en
    los atributos \texttt{ca} y \texttt{thal}. La distribución de la variable
    objetivo muestra que de los 303 pacientes, 164 no presentaban enfermedad
    cardiovascular (valor 0), mientras que 139 presentaban diversos grados de
    enfermedad (valores 1-4), esto indica que las clases son desbalanceadas.

    Entre las limitaciones del dataset se debe considerar su contexto temporal
    (datos de 1979-1988), donde los criterios diagnósticos y prácticas clínicas
    pueden diferir de los estándares actuales. Además, la muestra representa
    una población específica de pacientes referidos para angiografía, lo que
    puede introducir sesgos de selección. Cabe resaltar que el dataset no
    incluye factores de riesgo modernos como proteína C reactiva o score cálcico
    coronario.
    
    %%%%%%%%%%%%%%%%%%%%%%%%%%%%%%%%%%%%%%%%%%%%%%%%%%%%%%%%%%%%%%%%%%%%%%%%%%%%
    %%%%%%%%%%%%%%%%%%%%%%%%%%% PREPROCESAMIENTO %%%%%%%%%%%%%%%%%%%%%%%%%%%%%%%
    %%%%%%%%%%%%%%%%%%%%%%%%%%%%%%%%%%%%%%%%%%%%%%%%%%%%%%%%%%%%%%%%%%%%%%%%%%%%

    \section{Preprocesamiento}
    El preprocesamiento tiene como finalidad transformar la base de datos
    \textit{Heart Disease} en una matriz de diseño adecuado para el modelado
    supervisado. En particular, buscamos, 
    \begin{itemize}
        \item[a)] homogenizar la escala de las variables numéricas para
        algoritmos sensibles a la magnitud, 
        \item[b)] separar correctamente predictores $X$ y respuesta $y$
        evitando fugas de información. 
    \end{itemize}

    \subsection{Preparación y limpieza}

    Como mencionamos anteriormente, las variables \texttt{ca} y \texttt{thal}
    presentan valores faltantes. A continuación, mostramos de forma grafica
    como se distribiuyen estos valores faltantes en el dataset y a su vez
    analizamos la posible presencia de outliers a través de boxplots.

    \begin{figure}[ht]
        \centering
        \includegraphics[width=0.4\textwidth]{./figures/ca_thal_nans.png}
        \includegraphics[width=0.4\textwidth]{./figures/ca_thal_boxplots.png}
        \caption{Valores faltantes y boxplots de \texttt{ca} y \texttt{thal}}
        \label{fig:ca_thal}
    \end{figure}

    A simple vista observamos que los datos faltes no siguen un patrón
    específico, por lo que optamos por eliminar las filas con valores faltantes,
    resultando en un total de 297 observaciones. En cuanto a los
    outliers, considerando tambien la siguiente tabla, no se observan valores
    extremos evidentes que justifiquen una eliminación, por lo que los
    conservamos para el analisis posterior.

    \begin{table}[ht]
        \centering
        \caption{Distribución de Variables Categóricas}
        \label{tab:distribucion_categorica}
        \begin{tabular}{lcc}
            \toprule
            \textbf{Variable} & \textbf{Valor} & \textbf{Frecuencia} \\
            \midrule
            \multirow{ca} 
                & 0.0 & 176 \\
                & 1.0 & 65 \\
                & 2.0 & 38 \\
                & 3.0 & 20 \\
            \midrule
            \multirow{thal}
                & 3.0 & 166 \\
                & 7.0 & 117 \\
                & 6.0 & 18 \\
            \bottomrule
        \end{tabular}
    \end{table}

    Por otro lado, se seleccionó \tetit{StandardScaler} para el preprocesamiento
    de los datos debido a su idoneidad en el contexto de variables biomédicas.
    Este método transforma las características restándoles la media y
    escalándolas por la desviación estándar, resultando en una distribución con
    media cero y varianza unitaria. Esta estandarización es crucial para
    algoritmos sensibles a la escala, como la regresión logística y las redes
    neuronales, ya que acelera la convergencia y asegura que los coeficientes
    o pesos se ajusten de manera equilibrada. A diferencia de MinMaxScaler, que
    es sensible a valores atípicos, StandardScaler preserva la forma original de
    la distribución y maneja mejor los rangos variables de las características
    (como colesterol: 126-564 y presión arterial: 94-200). Dado que los valores
    atípicos en datos biomédicos suelen ser clínicamente relevantes y no errores
    de medición, StandardScaler resulta el balance óptimo entre robustez y
    preservación de la información.


    %%%%%%%%%%%%%%%%%%%%%%%%%%%%%%%%%%%%%%%%%%%%%%%%%%%%%%%%%%%%%%%%%%%%%%%%%%%%
    %%%%%%%%%%%%%%%%%%%%%%%%%%%%%%%%% DESARROLLO %%%%%%%%%%%%%%%%%%%%%%%%%%%%%%%
    %%%%%%%%%%%%%%%%%%%%%%%%%%%%%%%%%%%%%%%%%%%%%%%%%%%%%%%%%%%%%%%%%%%%%%%%%%%%


    \section{Clasificación y evaluación de modelos}
    
    Basados en las propiedades metodológicas y las ventajas prácticas que cada
    clasificador ofrece para el análisis de riesgo de enfermedades
    cardiovasculares, hemos elejido implementar tres clasificadores. Estos son:
    bosques aleatorios, regresión logística y redes neuronales.

    \subsection{Random Forest}
    
    El modelo \textit{Random Forest} combina múltiples árboles de decisión para
    mejorar la precisión y estabilidad del clasificador. Es adecuado para datos
    clínicos porque maneja variables categóricas y continuas sin
    preprocesamientos complejos, captura relaciones no lineales y ofrece medidas
    de importancia de variables que apoyan la interpretación.

    \begin{figure}[ht]
        \centering
        \includegraphics[width=0.5\textwidth]{./figures/feature_importance.png}
        \caption{Importancia de características en Random Forest}
        \label{fig:random_forest}
    \end{figure}

    \subsection{Regresión Logística}
    
    La regresión logística es un modelo lineal ampliamente utilizado en
    clasificación binaria por su sencillez e interpretabilidad. Proporciona
    coeficientes en términos de \textit{odds ratios}, genera probabilidades
    predichas adecuadas para definir umbrales clínicos y presenta bajo riesgo de
    sobreajuste, lo que la convierte en una línea base robusta. Su principal
    limitación es la suposición de linealidad en el logit.

    La importancia de las variables según los coeficientes del modelo se muestra
    en la siguiente figura:

    \begin{figure}[ht]
        \centering
        \includegraphics[width=0.5\textwidth]{./figures/logistic_coefficients.png}
        \caption{Coeficientes de la Regresión Logística}
        \label{fig:logistic_regression}
    \end{figure}

    \subsection{Red Neuronal}
    
    Las redes neuronales son modelos flexibles capaces de aprender patrones no
    lineales complejos y adaptarse mediante arquitecturas específicas y técnicas
    de regularización como \textit{dropout}. Resultan útiles cuando existen
    interacciones difíciles de capturar con modelos tradicionales. Sin embargo,
    requieren más datos y cuidado en su ajuste debido al riesgo de sobreajuste
    y su menor interpretabilidad.

    Durante el entrenamiento, se monitoreó la función de pérdida para evaluar la
    convergencia del modelo, como se observa en la siguiente figura:

    \begin{figure}[ht]
        \centering
        \includegraphics[width=0.5\textwidth]{./figures/loss_function.png}
        \caption{Curva de pérdida durante el entrenamiento de la Red Neuronal}
        \label{fig:nn_loss_curve}
    \end{figure}


    \subsection{Comparación de los modelos}

    Las gráficas de importancia de características en el bosque aleatorio
    evidencia qué variables aportan mayor poder predictivo. Por otro lado, los
    coeficientes de la regresión logística permiten entender la dirección e
    influencia de cada variable en el riesgo estimado. La curva de pérdida de la
    red neuronal confirma un proceso de aprendizaje estable sin señales
    evidentes de sobreajuste.

    Mostramos las siguientes mmétricas que se obtuvieron en cada modelo:
    \begin{table}[ht]
        \centering
        \caption{Métricas de desempeño para los clasificadores}
        \label{tab:metrics_cardiovascular}
        \begin{tabular}{lccc}
            \toprule
            \textbf{Métrica} & \textbf{Random Forest} & \textbf{Logistic Regression} & \textbf{Neural Network} \\
            \midrule
            Recall (Sensitivity)    & 0.750 & 0.786 & 0.857 \\
            Accuracy                & 0.817 & 0.833 & 0.850 \\
            Balanced Accuracy       & 0.812 & 0.830 & 0.850 \\
            AUC-ROC                 & 0.941 & 0.950 & 0.910 \\
            Precision               & 0.840 & 0.846 & 0.828 \\
            F1-score                & 0.792 & 0.815 & 0.842 \\
            \bottomrule
        \end{tabular}
    \end{table}

    Los tres modelos muestran un desempeño sólido, pero con diferencias
    relevantes según la métrica. La red neuronal alcanza la mayor recall (0.857)
    y la mayor accuracy y balanced accuracy (0.850), lo que indica que es la
    opción más efectiva para detectar casos positivos (minimizar falsos negativos),
    aspecto crítico en un problema clínico donde no detectar a un paciente
    enfermo es costoso. Sin embargo, su AUC-ROC (0.910) es la más baja de los
    tres, lo que sugiere que, globalmente, discrimina algo peor entre clases a
    través de todos los umbrales que la regresión logística (AUC 0.950). La 
    regresión logística presenta el mejor AUC-ROC y buena precisión (0.846) y F1
    (0.815), lo que la convierte en un modelo muy equilibrado y con excelente
    capacidad de discriminación (útil si se busca un compromiso entre detectar
    enfermos y limitar falsos positivos). El bosque aleatorio queda en posición
    intermedia: buena AUC (0.941) y precisión (0.840), pero menor recall
    (0.750), por lo que produce más falsos negativos que la red neuronal.
    
    Por lo tanto, si la prioridad clínica es maximizar la detección
    (minimizar FN) es conveniente usar la red neuronal; si se prioriza la
    capacidad de discriminación global y estabilidad interpretativa, la
    regresión logística es la opción preferible; el bosque aleatorio es una
    alternativa intermedia.
    
    A continuación, se presentan las matrices de confusión para cada modelo:

    \begin{figure}[ht]
        \centering
        \includegraphics[width=0.5\textwidth]{./figures/confusion_matrix.png}
        \caption{Matrices de confusión para los clasificadores}
        \label{fig:confusion_matrices}
    \end{figure}

    Las matrices de confusión ilustran claramente la diferencia en patrones de
    error entre modelos, destacando el mejor desempeño de la red neuronal en la
    detección de casos positivos.
    



    %%%%%%%%%%%%%%%%%%%%%%%%%%%%%%%%%%%%%%%%%%%%%%%%%%%%%%%%%%%%%%%%%%%%%%%%%%%%
    %%%%%%%%%%%%%%%%%%%%%%%%%%%%%%% CONCLUSIONES %%%%%%%%%%%%%%%%%%%%%%%%%%%%%%%
    %%%%%%%%%%%%%%%%%%%%%%%%%%%%%%%%%%%%%%%%%%%%%%%%%%%%%%%%%%%%%%%%%%%%%%%%%%%%


    \section{Conclusiones}

    El análisis desarrollado sobre el conjunto de datos clínicos permitió
    evaluar el desempeño de tres modelos de clasificación orientados a la
    detección de riesgo cardiovascular, identificando fortalezas, limitaciones y
    su utilidad potencial en un contexto aplicado. Los resultados muestran que
    la regresión logística ofrece la mayor capacidad discriminativa global con
    una AUC-ROC cercana a $0.95$, lo que la convierte en el modelo más estable
    cuando se requiere equilibrio entre sensibilidad y especificidad. Por su
    parte, la red neuronal presenta la sensibilidad más alta del estudio y
    alcanza la mejor exactitud y balanced accuracy, lo cual la posiciona como
    la alternativa más adecuada cuando la prioridad clínica es reducir al mínimo
    los falsos negativos y asegurar que la mayoría de los pacientes con riesgo
    sean identificados oportunamente. El modelo de bosques aleatorio, aunque
    competitivo, tiende a generar un mayor número de falsos negativos respecto
    a la red neuronal, situándose en un punto intermedio tanto en desempeño
    como en robustez.

    No obstante, el estudio presenta limitaciones relevantes: el conjunto de
    datos es antiguo y no necesariamente representativo de poblaciones actuales,
    existe presencia de valores faltantes y cierto desbalance de clases, y los
    datos provienen de pacientes referidos para procedimientos
    cardiovasculares, lo que puede introducir sesgos de selección.

    \section*{Referencias}

    % \printbibliography

    \begin{itemize}
        \item \textbf{Dataset:} Detrano, R. et al. (1988).
        \textit{Heart Disease Dataset}. UCI Machine Learning Repository. Recuperado el 25 de noviembre de 2025 de https://archive.ics.uci.edu/ml/datasets/Heart+Disease
        
        \item \textbf{Organización Mundial de la Salud (2021).}
        \textit{Cardiovascular Diseases (CVDs)}. Recuperado de https://www.who.
        int/news-room/fact-sheets/detail/cardiovascular-diseases-(cvds). Nota: Las ECV representan el 32\% de las muertes globales
    \end{itemize}

\end{document}
