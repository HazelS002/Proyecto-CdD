\documentclass[10pt]{article}
\usepackage[spanish]{babel}
\usepackage[a4paper, tmargin=0.75in, lmargin=0.80in, rmargin=0.80in,
    bmargin=1in]{geometry}
\usepackage{hyperref}
%\usepackage{multicol}
\hypersetup{
    colorlinks=true,
    linkcolor=black,
    filecolor=magenta,      
    urlcolor=blue,
    citecolor=black,
}
%\usepackage[numbers,sort&compress]{natbib} % for a numerical citation list
\usepackage{natbib} % to cite references by surname and year
\usepackage{graphicx}
\usepackage{amsfonts}
\usepackage{amsthm}
\usepackage{amssymb}
\usepackage{lipsum}
\usepackage{amsmath}
\usepackage{tabularx}
\usepackage{pdflscape}
\usepackage{booktabs}
\usepackage{bbm}
\usepackage{listings}
\usepackage{xcolor}

\usepackage[
  backend=biber,
  style=alphabetic,
  citestyle=alphabetic,
  doi=true,
  url=true,
  isbn=false,
  eprint=false,
  maxbibnames=99
]{biblatex}

\addbibresource{referencias.bib}

% Definimos colores estilo "terminal con fondo negro"
\definecolor{backcolour}{rgb}{0.1,0.1,0.1}
\definecolor{codegreen}{rgb}{0,0.8,0}
\definecolor{codegray}{rgb}{0.7,0.7,0.7}
\definecolor{codepurple}{rgb}{0.8,0.6,1}
\definecolor{codewhite}{rgb}{1,1,1}


\lstdefinestyle{mypython}{
    backgroundcolor=\color{backcolour},
    basicstyle=\ttfamily\scriptsize\color{codewhite},
    commentstyle=\color{codegreen},
    keywordstyle=\color{codepurple},
    stringstyle=\color{codegreen},
    numbers=left,
    numberstyle=\tiny\color{codegray},
    breaklines=true,
    breakatwhitespace=false,
    showstringspaces=false,
    tabsize=4
}

\lstset{style=mypython}


\pagestyle{empty}


%%%%%%%%%%%%%%%%%%%%%%%%%%%%%%%%%%%%%%%%%%%%%%%%%%
%%%%%%%%%%%%%%%%%%%%%%%%%%%%%%%%%%%%%%%%%%%%%%%%%%
% ENTER SOME IMPORTANT INFORMATION
%%%%%%%%%%%%%%%%%%%%%%%%%%%%%%%%%%%%%%%%%%%%%%%%%%
%%%%%%%%%%%%%%%%%%%%%%%%%%%%%%%%%%%%%%%%%%%%%%%%%%
\newcommand{\studentname}{Hazel Shamed Sánchez Chávez}
% \newcommand{\researchcentre}{Maestría en Probabilidad y Estadística}
\newcommand{\institution}{Centro de Investigación en Matemáticas (CIMAT)}
\newcommand{\projecttitle}{Proyecto: Clasificación de Riesgo Cardiovascular}
\newcommand{\supervisor}{Dr. Marco Antonio Aquino López}
%%%%%%%%%%%%%%%%%%%%%%%%%%%%%%%%%%%%%%%%%%%%%%%%%%
%%%%%%%%%%%%%%%%%%%%%%%%%%%%%%%%%%%%%%%%%%%%%%%%%%

\setlength{\parindent}{1em}
\setlength{\parskip}{1em}


\begin{document}

    \begin{center}
        {\Large{Proyecto 2: Clasificación de Riesgo Cardiovascular}} \\
        \vspace{2mm}
        {\Large{Introducción a la Ciencia de Datos}} \\
    \end{center}

    \vspace{5mm}
    \hrule
    \vspace{1mm}
    \hrule

    \vspace{3mm}
    \begin{tabular}{ll} 
        Estudiante:           	        & {\studentname}   \\ 
        % Programa Educativo: 	        & {\researchcentre}  \\ 
        Institución:                 & {\institution}  \\
        Profesor: 	                 & {\supervisor}  \\ 
    \end{tabular}

    \vspace{3mm}
    \hrule
    \vspace{1mm}
    \hrule

    % \begin{abstract}
    % \end{abstract}

    %%%%%%%%%%%%%%%%%%%%%%%%%%%%%%%%%%%%%%%%%%%%%%%%%%%%%%%%%%%%%%%%%%%%%%%%%%%%
    %%%%%%%%%%%%%%%%%%%%%%%%%%% INTRODUCCIÓN %%%%%%%%%%%%%%%%%%%%%%%%%%%%%%%
    %%%%%%%%%%%%%%%%%%%%%%%%%%%%%%%%%%%%%%%%%%%%%%%%%%%%%%%%%%%%%%%%%%%%%%%%%%%%

    \section{Introducción}

    En el ámbito de la salud pública, las enfermedades
    cardiovasculares (ECV) constituyen una de las principales causas de
    mortalidad a nivel global, representando aproximadamente el 32\% de todas
    las defunciones según la Organización Mundial de la Salud. La identificación
    temprana de individuos en riesgo representa un desafío crítico para los
    sistemas de salud, con implicaciones significativas en la reducción de la
    carga asistencial y la mejora de resultados clínicos.

    Este proyecto se enfoca en el desarrollo de un modelo predictivo de
    clasificación de riesgo cardiovascular utilizando el dataset
    \textit{"Heart Disease"} del repositorio UCI Machine Learning, que contiene
    registros clínicos y biomédicos de 303 pacientes. La base de datos integra
    signos clínicos, parámetros de laboratorio y características
    electrocardiográficas, proporcionando una base multidimensional para el
    análisis.

    Las ECV representan condiciones patológicas que afectan el corazón y los
    vasos sanguíneos, incluyendo enfermedad coronaria, enfermedad
    cerebrovascular y cardiopatía reumática. La naturaleza multifactorial de
    estas patologías demanda aproximaciones analíticas capaces de integrar
    diversos predictores, desde factores de riesgo convencionales hasta
    marcadores específicos de función cardíaca.

    El objetivo central de este proyecto es construir un clasificador
    binario que permita discriminar entre pacientes con presencia y ausencia de
    enfermedad cardiovascular, utilizando únicamente variables accesibles
    mediante exámenes clínicos rutinarios. Este enfoque pretende superar las
    limitaciones de los métodos diagnósticos invasivos, optimizando la
    asignación de recursos diagnósticos especializados.

    La implementación exitosa de este modelo podría servir como herramienta de
    apoyo a la decisión clínica, facilitando la estratificación prioritaria de
    pacientes y contribuyendo a estrategias de prevención secundaria más
    efectivas en el manejo de enfermedades cardiovasculares.


    \section{Exploración inicial de los datos}\label{sec:dataset}

    \subsection{Base de datos \textit{Heart Disease}.}


    El dataset ``Heart Disease'', de $303$ filas por $13$ columnas (mas etiquetas),
    fue recopilado principalmente por la \textbf{Cleveland Clinic Foundation}
    bajo la dirección del Dr. Robert Detrano, M.D., Ph.D., con la colaboración
    de múltiples instituciones internacionales incluyendo el Hungarian Institute
    of Cardiology de Budapest, University Hospital de Zurich y University
    Hospital de Basel. Los datos fueron donados al UCI Machine Learning
    Repository en 1988, donde están disponibles públicamente con el
    identificador $45$.

    El estudio se condujo como una investigación observacional multicéntrica
    entre 1979 y 1988, abarcando aproximadamente 10 años de recolección de
    datos. La población del estudio consistió en 303 pacientes referidos para
    evaluación coronaria en las instituciones participantes. El criterio de
    referencia para el diagnóstico fue la angiografía coronaria, definiendo como
    caso positivo la presencia de $\geq$50\% de estrechamiento diametral.

    El dataset original contenía 76 variables recolectadas, pero para el
    análisis estándar se seleccionaron 14 atributos que incluyen signos
    clínicos, parámetros de laboratorio y características
    electrocardiográficas.
  
    \subsection{Variables del Dataset}

    Al descargar el dataset, las variables categoricas ya vienen dadas de forma
    numerica, asignando un valor entero a cada categoría de cada variable. Esto
    facilita la manipulación y calculos de los datos. Presentamos una tabla
    resumen de las variables del dataset:

    \begin{table}[ht]
        \centering
        \small
        \caption{Variables del Dataset Heart Disease}
        \label{tab:variables}
        \begin{tabular}{>{\raggedright\arraybackslash}p{2cm}>{\raggedright\arraybackslash}p{2cm}>{\raggedright\arraybackslash}p{6cm}>{\raggedright\arraybackslash}p{3cm}}
            \toprule
            \textbf{Atributo} & \textbf{Tipo} & \textbf{Descripción} & \textbf{Valores} \\
            \midrule
            age & Continuo & Edad en años & 29-77 \\
            sex & Categórico & Sexo & 0 = fem, 1 = masc \\
            cp & Categórico & Tipo de dolor torácico & 1-4 \\
            trestbps & Continuo & Presión arterial en reposo (mm Hg) & 94-200 \\
            chol & Continuo & Colesterol sérico (mg/dl) & 126-564 \\
            fbs & Categórico & Glicemia en ayunas $>$120 mg/dl & 0 = no, 1 = sí \\
            restecg & Categórico & Resultado ECG en reposo & 0-2 \\
            thalach & Continuo & Frecuencia cardíaca máxima alcanzada & 71-202 \\
            exang & Categórico & Angina inducida por ejercicio & 0 = no, 1 = sí \\
            oldpeak & Continuo & Depresión ST inducida por ejercicio & 0-6.2 \\
            slope & Categórico & Pendiente del segmento ST en ejercicio & 1-3 \\
            ca & Continuo & Número de vasos principales coloreados & 0-3 \\
            thal & Categórico & Tipo de defecto talio & 3 = normal, 6 = defecto, 7 = reversible \\
            \midrule
            num & Objetivo & Diagnóstico de enfermedad & 0 = no, 1-4 = sí \\
            \bottomrule
        \end{tabular}
    \end{table}

    Estos signos clínicos, parámetros de laboratorio y características
    electrocardiográficas se interpretan desde el punto de vista médico de la
    siguiente manera:
    \begin{itemize}
        \item El dolor torácico (cp) se clasifica en cuatro tipos: anginoso
        típico (dolor opresivo por esfuerzo), atípico, no anginoso y ausente.
        \item El colesterol sérico (chol) cuantifica los lípidos en sangre en
        $mg/dl$.
        \item La glicemia en ayunas (fbs) indica diabetes si supera $120 mg/dl$.
        \item El electrocardiograma en reposo (restecg) detecta anomalías ST-T o
        hipertrofia ventricular.
        \item La frecuencia cardíaca máxima (thalach) registra el pico de
        esfuerzo.
        \item La angina inducida por ejercicio (exang) señala dolor durante la
        prueba de esfuerzo.
        \item La depresión del segmento ST (oldpeak) mide en milivolts la
        isquemia miocárdica durante el ejercicio.
        \item La pendiente del segmento ST (slope) evalúa si es ascendente
        (normal), plana o descendente (patológica).
        \item Los vasos principales coloreados (ca) indican cuántas arterias
        coronarias tienen obstrucción ≥50\% en angiografía.
        \item La prueba de talio (thal) evalúa perfusión cardíaca: normal,
        defecto fijo (infarto) o reversible (isquemia).
        \item El diagnóstico (num) clasifica la severidad de la enfermedad
        desde $0$ (ausente) hasta $4$ (severa).
    \end{itemize}
    
    A continuación, se presentan las estadísticas descriptivas de estas 
    variables:

    \newpage

    \begin{table}[ht]
        \centering
        \small
        \caption{Estadísticas Descriptivas del Dataset}
        \label{tab:estadisticas_descriptivas}
        \begin{tabular}{l*{13}{r}}
            \toprule
            & \textbf{age} & \textbf{sex} & \textbf{cp} & \textbf{trestbps} & \textbf{chol} & \textbf{fbs} & \textbf{restecg} & \textbf{thalach} & \textbf{exang} & \textbf{oldpeak} & \textbf{slope} & \textbf{ca} & \textbf{thal} \\
            \midrule
            count & 303 & 303 & 303 & 303 & 303 & 303 & 303 & 303 & 303 & 303 & 303 & 299 & 301 \\
            mean & 54.4 & 0.68 & 3.16 & 131.69 & 246.69 & 0.15 & 0.99 & 149.61 & 0.33 & 1.04 & 1.60 & 0.67 & 4.73 \\
            std & 9.04 & 0.47 & 0.96 & 17.60 & 51.78 & 0.36 & 0.99 & 22.88 & 0.47 & 1.16 & 0.62 & 0.94 & 1.94 \\
            min & 29 & 0 & 1 & 94 & 126 & 0 & 0 & 71 & 0 & 0.00 & 1 & 0 & 3 \\
            25\% & 48 & 0 & 3 & 120 & 211 & 0 & 0 & 134 & 0 & 0.00 & 1 & 0 & 3 \\
            50\% & 56 & 1 & 3 & 130 & 241 & 0 & 1 & 153 & 0 & 0.80 & 2 & 0 & 3 \\
            75\% & 61 & 1 & 4 & 140 & 275 & 0 & 2 & 166 & 1 & 1.60 & 2 & 1 & 7 \\
            max & 77 & 1 & 4 & 200 & 564 & 1 & 2 & 202 & 1 & 6.20 & 3 & 3 & 7 \\
            \bottomrule
        \end{tabular}
    \end{table}

    \subsection{Consideraciones Adicionales}

    Es importante señalar que el dataset presenta algunos valores faltantes en
    los atributos \texttt{ca} y \texttt{thal}. La distribución de la variable
    objetivo muestra que de los 303 pacientes, 164 no presentaban enfermedad
    cardiovascular (valor 0), mientras que 139 presentaban diversos grados de
    enfermedad (valores 1-4), esto indica que las clases son desbalanceadas.

    Entre las limitaciones del dataset se debe considerar su contexto temporal
    (datos de 1979-1988), donde los criterios diagnósticos y prácticas clínicas
    pueden diferir de los estándares actuales. Además, la muestra representa
    una población específica de pacientes referidos para angiografía, lo que
    puede introducir sesgos de selección. Cabe resaltar que el dataset no
    incluye factores de riesgo modernos como proteína C reactiva o score cálcico
    coronario.
    
    %%%%%%%%%%%%%%%%%%%%%%%%%%%%%%%%%%%%%%%%%%%%%%%%%%%%%%%%%%%%%%%%%%%%%%%%%%%%
    %%%%%%%%%%%%%%%%%%%%%%%%%%% PREPROCESAMIENTO %%%%%%%%%%%%%%%%%%%%%%%%%%%%%%%
    %%%%%%%%%%%%%%%%%%%%%%%%%%%%%%%%%%%%%%%%%%%%%%%%%%%%%%%%%%%%%%%%%%%%%%%%%%%%

    \section{Preprocesamiento}
    El preprocesamiento tiene como finalidad transformar la base de datos
    \textit{Heart Disease} en una matriz de diseño adecuado para el modelado
    supervisado. En particular, buscamos, 
    \begin{itemize}
        \item[a)] homogenizar la escala de las variables numéricas para
        algoritmos sensibles a la magnitud, 
        \item[b)] separar correctamente predictores $X$ y respuesta $y$
        evitando fugas de información. 
    \end{itemize}

    \subsection{Preparación y limpieza}

    Como mencionamos anteriormente, las variables \texttt{ca} y \texttt{thal}
    presentan valores faltantes. A continuación, mostramos de forma grafica
    como se distribiuyen estos valores faltantes en el dataset y a su vez
    analizamos la posible presencia de outliers a través de boxplots.

    \begin{figure}[ht]
        \centering
        \includegraphics[width=0.4\textwidth]{./figures/ca_thal_nans.png}
        \includegraphics[width=0.4\textwidth]{./figures/ca_thal_boxplots.png}
        \caption{Valores faltantes y boxplots de \texttt{ca} y \texttt{thal}}
        \label{fig:ca_thal}
    \end{figure}

    A simple vista observamos que los datos faltes no siguen un patrón
    específico, por lo que optamos por eliminar las filas con valores faltantes,
    resultando en un total de 297 observaciones. En cuanto a los
    outliers, considerando tambien la siguiente tabla, no se observan valores
    extremos evidentes que justifiquen una eliminación, por lo que los
    conservamos para el analisis posterior.

    \begin{table}[ht]
        \centering
        \caption{Distribución de Variables Categóricas}
        \label{tab:distribucion_categorica}
        \begin{tabular}{lcc}
            \toprule
            \textbf{Variable} & \textbf{Valor} & \textbf{Frecuencia} \\
            \midrule
            \multirow{ca} 
                & 0.0 & 176 \\
                & 1.0 & 65 \\
                & 2.0 & 38 \\
                & 3.0 & 20 \\
            \midrule
            \multirow{thal}
                & 3.0 & 166 \\
                & 7.0 & 117 \\
                & 6.0 & 18 \\
            \bottomrule
        \end{tabular}
    \end{table}

    Por otro lado, se seleccionó \tetit{StandardScaler} para el preprocesamiento
    de los datos debido a su idoneidad en el contexto de variables biomédicas.
    Este método transforma las características restándoles la media y
    escalándolas por la desviación estándar, resultando en una distribución con
    media cero y varianza unitaria. Esta estandarización es crucial para
    algoritmos sensibles a la escala, como la regresión logística y las redes
    neuronales, ya que acelera la convergencia y asegura que los coeficientes
    o pesos se ajusten de manera equilibrada. A diferencia de MinMaxScaler, que
    es sensible a valores atípicos, StandardScaler preserva la forma original de
    la distribución y maneja mejor los rangos variables de las características
    (como colesterol: 126-564 y presión arterial: 94-200). Dado que los valores
    atípicos en datos biomédicos suelen ser clínicamente relevantes y no errores
    de medición, StandardScaler resulta el balance óptimo entre robustez y
    preservación de la información.


    %%%%%%%%%%%%%%%%%%%%%%%%%%%%%%%%%%%%%%%%%%%%%%%%%%%%%%%%%%%%%%%%%%%%%%%%%%%%
    %%%%%%%%%%%%%%%%%%%%%%%%%%%%%%%%% DESARROLLO %%%%%%%%%%%%%%%%%%%%%%%%%%%%%%%
    %%%%%%%%%%%%%%%%%%%%%%%%%%%%%%%%%%%%%%%%%%%%%%%%%%%%%%%%%%%%%%%%%%%%%%%%%%%%


    \section{Clasificación y evaluación de modelos}


    %%%%%%%%%%%%%%%%%%%%%%%%%%%%%%%%%%%%%%%%%%%%%%%%%%%%%%%%%%%%%%%%%%%%%%%%%%%%
    %%%%%%%%%%%%%%%%%%%%%%%%%%%%%%% CONCLUSIONES %%%%%%%%%%%%%%%%%%%%%%%%%%%%%%%
    %%%%%%%%%%%%%%%%%%%%%%%%%%%%%%%%%%%%%%%%%%%%%%%%%%%%%%%%%%%%%%%%%%%%%%%%%%%%


    \section{Conclusiones}



\end{document}


@misc{heart_disease_1988,
  title={Heart Disease Data Set},
  author={Detrano, R. and Janosi, A. and Steinbrunn, W. and Pfisterer, M.},
  year={1988},
  publisher={UCI Machine Learning Repository}
}


@article{detrano1989international,
  title={International application of a new probability algorithm for the diagnosis of coronary artery disease},
  author={Detrano, R. and others},
  journal={The American Journal of Cardiology},
  volume={64},
  number={5},
  pages={304--310},
  year={1989},
  publisher={Elsevier}
}